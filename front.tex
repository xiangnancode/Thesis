%
%  revised  front.tex  2017-01-08  Mark Senn  http://engineering.purdue.edu/~mark
%  created  front.tex  2003-06-02  Mark Senn  http://engineering.purdue.edu/~mark
%
%  This is ``front matter'' for the thesis.
%
%  Regarding ``References'' below:
%      KEY    MEANING
%      PU     ``A Manual for the Preparation of Graduate Theses'',
%             The Graduate School, Purdue University, 1996.
%      PU8    ``A Manual for the Preparation of Graduate Theses'',
%             Eighth Revise Edition, Purdue University.
%      TCMOS  The Chicago Manual of Style, Edition 14.
%      WNNCD  Webster's Ninth New Collegiate Dictionary.
%
%  Lines marked with "%%" may need to be changed.
%

  % Statement of Thesis/Dissertation Approval Page
  % This page is REQUIRED.  The page should be numbered page ``ii''
  % and should NOT be listed in your TABLE OF CONTENTS.
  % References: PU8 ordinal pages 5 and 29.
  % The web page https://engineering.purdue.edu/AAE retrieved on
  % January 8, 2017 had "School of Aeronautics and Astronautics"---that
  % is used instead of "Department af Aeronautics and Astronautics"
  % below.
\begin{statement}
  \entry{Dr.~Lingxi Li, Chair}{Department of Electrical and Computer Engineering}
  \entry{Dr.~Brian King}{Department of Electrical and Computer Engineering}
  \entry{Dr.~Maher Rizkalla}{Department of Electrical and Computer Engineering}
  \approvedby{Dr.~Brian King}{Head of the Electrical and Computer Engineering Graduate Program}
\end{statement}

  % Dedication page is optional.
  % A name and often a message in tribute to a person or cause.
  % References: PU 15, WNNCD 332.
%\begin{dedication}
%  This is the dedication.
%\end{dedication}

  % Acknowledgements page is optional but most theses include
  % a brief statement of apreciation or recognition of special
  % assistance.
  % Reference: PU 16.
\begin{acknowledgments}
  I would like to thank my advisor for his guidance and encouragement through the research and writing process. I am grateful to all of my committee members for the time and energy they have put into helping me complete my research. 
	
	\bigskip
	My family and friends kept me motivated and happy during this long process.
	
	\bigskip
	Your support means so much to me. Thank you.
\end{acknowledgments}

  % The preface is optional.
  % References: PU 16, TCMOS 1.49, WNNCD 927.
%\begin{preface}
%  This is the preface.
%\end{preface}

  % The Table of Contents is required.
  % The Table of Contents will be automatically created for you
  % using information you supply in
  %     \chapter
  %     \section
  %     \subsection
  %     \subsubsection
  % commands.
  % Reference: PU 16.
\tableofcontents

  % If your thesis has tables, a list of tables is required.
  % The List of Tables will be automatically created for you using
  % information you supply in
  %     \begin{table} ... \end{table}
  % environments.
  % Reference: PU 16.
\listoftables

  % If your thesis has figures, a list of figures is required.
  % The List of Figures will be automatically created for you using
  % information you supply in
  %     \begin{figure} ... \end{figure}
  % environments.
  % Reference: PU 16.
\listoffigures

  % List of Symbols is optional.
  % Reference: PU 17.
%\begin{symbols}
%  $m$& mass\cr
%  $v$& velocity\cr
%\end{symbols}

  % List of Abbreviations is optional.
  % Reference: PU 17.
%\begin{abbreviations}
%  abbr& abbreviation\cr
%  bcf& billion cubic feet\cr
%  BMOC& big man on campus\cr
%\end{abbreviations}

  % Nomenclature is optional.
  % Reference: PU 17.
%\begin{nomenclature}
%  Alanine& 2-Aminopropanoic acid\cr
%  Valine& 2-Amino-3-methylbutanoic acid\cr
%\end{nomenclature}

  % Glossary is optional
  % Reference: PU 17.
%\begin{glossary}
%  chick& female, usually young\cr
%  dude& male, usually young\cr
%\end{glossary}

  % Abstract is required.
  % Note that the information for the first paragraph of the output
  % doesn't need to be input here...it is put in automatically from
  % information you supplied earlier using \title, \author, \degree,
  % and \majorprof.
  % Reference: PU 17.
\begin{abstract}
  Currently, there are many types of Automatic Guided Vehicles (AGVs) in different industries. Typically their job is to move raw materials or parts around a manufacturing facility, and they can be very accurate by following the guides from wires in the floor, magnets, laser, or vision. However, currently AGVs only work indoors. Therefore, the purpose of this thesis is to discuss the implementation of the outdoor AGV. An outdoor AGV has much more constraints than an indoor one. The environment indoors can be easily controlled while the outdoor cannot because there could be such problems as rough outdoor surfaces, no pre-set guiding wires or magnets, vision blocking by dust, and so on. The solution, which will be introduced in this paper, to achieve the outdoor AGV is laser guidance. In addition, a buffer will be installed to stabilize the cargo or others working devices, to prevent them from the shaking due to the rough outdoor surfaces. To be more specific, a prototype will be built to simulate the working of a seeder. In agriculture, it is very important to plant corns in a straight line, not only to increase the absorption of sunlight and ventilation, but also to reduce the work of irrigation, fertilizing, and harvest. Furthermore, to achieve unmanned agriculture, a corn field with straight lines will also be a good condition for other agriculture robots.
\end{abstract}
