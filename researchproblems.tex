\chapter{RESEARCH PROBLEMS}
Although some cites in China, such as Beijing and Shanghai, are well-known magnificent modern cities and the GDP of China was ranked number two in the world in 2015 \cite{GDP2015}, agriculture is a short plate of China's modernization. According to the current situation in China, Chinese farmers are unable to afford neither big farm machinery, nor precision GPS guidance systems. Since the arable land and annual income is only about 0.3$hm^{2}$ and 7916.6RMB for each farmer, small and cheap farm machinery is the best choice. A common farm tractor in the United States is normally 50-75 horsepower, while the popular ones in China are only about 20 horsepower. The price for a small tractor with 20 horsepower is about 13,000RMB (about 1950USD), which is affordable for one family or several together. However, the cost for an agricultural GPS guidance systems is too much; farmers in China cannot afford them. The WAAS GPS offers the lowest cost, but 2,000USD is enough for another small tractor. (Table 3.1) In addition, WAAS only works in the United States. These differences between the China and the United States mean that it is infeasible to develop agriculture in China the same way that it is in the United States. Therefore, finding an accurate and low-cost alternative solution for planting crops in parallel rows is the first step to achieve the modern farming and precision agriculture in China.

\begin{table}[ht!]
\caption{GPS Guidance Systems and Prices \cite{PriceR}}
\begin{center}	
\begin{tabular}{|l|l|l|}
\hline
NAME & ACCURACY & Estimated Price \\ 
\hline
WAAS & 12 to 15 inches & \$2,000 \\
\hline
WAAS with AutoSteer &  12 to 15 inches & \$4,000\\
\hline
OmniStar & sub-meter & \$6,000 + \$1,150 per year\\
\hline
Radio Beacon & - & \$12,500\\
\hline
RTK & 1 inch & \$18,500 + \$1,125 per year\\
\hline

\end{tabular}
\end{center}					
\end{table}

Based on the related researches and the situations in China, achieving precision agriculture means not only improving the accuracy of the trajectory of farm machinery, but also lowering the cost of guidance systems. Therefore, the research problem of this paper is to find a low-cost solution that improves the accuracy of a tractor's trajectory. The idea of reducing cost is to implement an add-on device that is compatible to any of the existing farm tractors, so that farmers can just pay a small amount of money to upgrade their own tractor. 

%Based on the related researches, achieving precision agriculture is not only to improve the accuracy of the position of farm machinery, but also to improve the accuracy of the trajectory of tractor. With the localization technology, correction could be made once the vehicle off the track; with the sliding estimation technology, correction could be made once the tire slid. All the researches have done were trying to find the exact position of the tractor or to estimate sliding of it then make the correction. However, making correction means errors already occurred. To prevent error from happening, an additional device and guide system was designed.