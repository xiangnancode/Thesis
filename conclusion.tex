\chapter{CONCLUSION}

\section{Introduction}
This paper introduced a new guidance system design for agricultural machinery. This design contains a buffer and a laser pointer. The buffer is a device that needs to be installed on a tractor attachment, and the laser pointer needs to be placed at the end of crop field rows facing to the back of the tractor. First, in order to use this guidance system, the laser must be put in the right position and face the right direction. Then the laser must be projected onto the curtain of the buffer. After the buffer has initialized, the tractor is ready for operation. This system is not only suitable for existing GPS-guided agricultural machinery, but it also works on machinery without any guidance systems on board in order to enhance the accuracy from $\pm$ 10 cm to $\pm$ 2 cm. The only difference is addition LEDs needs to be installed on the dashboard of the tractor. If the driver can drive at a low speed and handle the steering wheel carefully based on the signal on the dashboard, which is given by the laser guidance system, it is expected to have the same $\pm$ 2 cm accuracy. 

%使用这个设备能达到那些效果。为农业育种提供方便,为未来的无人农业提供先决条件。

\section{Technical Highlights}

\subsection{Local Laser Reference}
The guidance system introduced in this paper uses a laser as a local reference. A local reference is able to provide a more reliable guide because there are less noise factors. For instance, GPS is based on the distance measurement between several satellites and a receiver, and the high atmosphere has a significant affect on its accuracy. 

\subsection{Buffer}
The method that how this laser guidance system corrects error is to install a buffer to the tractor attachment. The method that current technology use to make corrections is to turn the steeling wheel of the tractor. This method is inefficient, because no matter how fast the correction can be made, there are deviations in the trajectory. However, deviations will never take place with the buffer installed on the tractor attachment. The buffer always stays at a relatively steady position with the laser reference. It allows the tractor to deviate for a little bit while keeping the trajectory on the right track before the tractor makes its correction.

\subsection{Image Processing}
The algorithm contains two parts. High-intensity detection is used when the tractor is close to the laser pointer. The laser pointer is designed to have range that is over 200 $m$, so it is very bright at a close distance. High-intensity detection provides a more stable detection because it is too bright to detect color. Once the color detection shows a stable result, high-intensity detection will be disabled. 

%利用摄像头来引导是趋势,摄像头成本低,易于改进,人就是靠视觉获取信息,同样机器也可以

\section{Feasibility}
The laser guidance system introduced in this paper is an add-on device to the current tractor attachments. Users does not need to buy new tractors or attachments, just the two parts of this device. Therefore, the cost is very low. However, the set-up procedure is a little bit complex. Therefore, it may not be the best choice to spread out to every farm because of the 76.2 $cm$ row spacing that is still mostly in use. The best place to apply this design is the experimental crop field, with the goal of finding the high performance breed. The seeding position is very strict, so that different breeds are comparable because of the identical growth environment of every plant. 

In the future, the row spacings will be as narrow as possible so that more crops can be planted, so high accuracy agriculture machinery is necessary for every corner of the world. The advantage of laser guidance systems is that every machinery on duty is able to upgrade. It is not necessary to buy a brand-new tractor that has a GPS guidance system on board and then upgrade based on the GPS. Therefore, laser guide system is a great choice for most of the developing country where GPS tractors are not popular. This is a very economical way for them to achieve the precision agriculture.


\section{Constraints}
The guidance system introduced in this paper is based on an image processing algorithm. Therefore, any condition that affects the camera will affect this system. The most common problem is dust caused by the running tractor. Heavy dust may block the laser light and cause the guidance system to lose reference. As for the laser light, a 5 $mW$ green laser pointer is in use for now and a higher power laser pointer may be used in the future to provide a farther distance reference. Therefore, safety is a big problem because the high-power laser can damage retinas at a close distance. Furthermore, it is very important to set up the laser pointer in the correct direction. A small error of angle will cause a big difference because of the long column distance. Based on these two reasons, it is necessary to have a trained operator. Last, because of limited resources and time, this guidance system is only designed for flat land. It will not work on sloping fields.


\section{Application for Other Areas}
Essentially, the design put forth by this paper is guidance system. Therefore, the laser guidance system can apply not only in agriculture machinery, but also in any other outdoor vehicles that need a precision guidance. In the aspect of agriculture, the buffer only stabilizes in a horizontal direction. In fact, the buffer can also stabilize in a vertical direction. This feature can be applied to a cart that moves fragile materials on a bumpy ground. For example, the outdoor AGVs with the laser guidance system are able to move fragile materials, such as glass, around the construction area just like indoor ones working in a factory. 


\section{Importance of Outdoor AGVs}
Outdoor AGVs was an essential and practical technology for the future of modern agriculture. Most of the current AGV are only is for indoor use, which is not applicable for outdoor environments. In outdoor conditions, various problems occur such as rough ground, unpredictable weather, and high expense of equipment, which is very different from indoor conditions. Furthermore, since most indoor AGV guidance systems require pre-set wires, magnets, or paints, it is very expensive and inconvenient to use them outdoors. Therefore, it is urgent to developed a feasible AGV for outdoor to use this paper mainly focuses on straight field rows. The significance of planting crops in a straight line is to produce the same condition for all areas in the experiment as well as to lay the foundation for further agriculture purposes; for example, straight rows reduce the work of irrigation by controlling the position of irrigation and creating convenience for agriculture robots. Hence, the development of laser guidance systems for AGVs can solve both problems mentioned and inspire more ideas for outdoor AGV researches.
%本文重点讨论了关于户外AGV在农业方面的应用,现在工厂有流水线,但是室外的农业却没有这么高效率。目前来说,传统的GPS农机或许在播种上能满足现在的需求。但是要满足育种试验田的播种需求,还是需要户外AGV。放眼未来,要实现想工厂流水线一样高效的无人农业,精准的播种是第一步也是非常重要的一步。

\section{Overview of Significance}
The laser guidance system introduced in this paper is an accurate and economic method to guide the outdoor vehicles. There are two devices in this guidance system; one of them is the laser pointer, and the other one is the buffer. The laser pointer emits a straight laser light that provides the reference for the guidance system. Its position is fixed on the ground and pointing to the direction that the vehicle needs to move. The buffer is installed on the vehicle to provide the guide. There is a camera on the buffer that is used to receive the laser light reference, and then the on board microprocessor analyzes the information by running the designed algorithm. Finally, the microprocessor sends action commands out. In this paper, these commands are sent to the buffer to stabilize the tractor attachment, as well as to the driver if there is no GPS guidance system on board. For other aspects, these commands not only guide the movement of the vehicle, but also can stabilize the cargo that was carried by the AGV. It is very useful if the AGV is transporting fragile materials on a dumpy ground.


\section{Limitations}
Laser light travels in a straight line, so this guidance system is only good at moving in a straight line. In order to make turns, it is better to have an additional guidance system on board, such as a GPS, to cooperate. Moreover, one laser pointer must to be set for each segment of the straight route. On the other hand, the laser guidance system will also be lost if the route is not monotonically up or down. Imagine a situation where a vehicle goes uphill and then downhill; the laser pointer can only guide the former movement. Therefore, there is a limitation in the geography of the working area.

The camera on the vehicle observes the laser light as a reference, but it might be hard to "see" the laser light under some situations. The observation will be affected by the surrounding lights. When this guidance system is used outdoors, the main factor that affects the laser light detection is strong sunlight. If the light intensity in the surroundings is too high, the laser light could be polluted so that green color cannot be detected. Therefore, it might be required to operate early in the morning or late in the afternoon, when the sun light is weak. 

\section{Further Improvement}
Based on the limitations, the laser guidance system can be improved in two ways in the future. First, the system could be adapted to diverse geographic working area. On the one hand, since some of crop fields are circular for the convenience of the giant irrigation sprinklers, it is better to plant the crops in the same shape. On the other hand, not all of the crop fields are flat. In fact, some farm land is in mountainous areas. It is necessary to make the guidance system able to work on every kind of farm. Second, the system could be adapted to reduce the affect from the strong sun light. Farming operations generally take place during the day. Furthermore, most of the crops require a very limited time window to be planted. In order to apply the laser guidance system widely, it would be more convenient if the guidance system were able to work properly at any time. 